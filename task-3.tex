\documentclass[10pt, a4paper]{article}
\usepackage[utf8]{inputenc}
\usepackage[english,russian]{babel}
\usepackage[colorlinks=true,linkcolor=blue]{hyperref}
\usepackage{listings}
\usepackage{color}
\usepackage{verbatim}
\usepackage{amsmath}

\usepackage{indentfirst}


\title{Задание практикума: Параллельная реализация решения уравнения теплопроводности на плоскости}
\author{Сальников А.Н.}
\date{}

\begin{document}

\maketitle

\section{Общее описание}

Требуется написать параллельную программу с применением технологий MPI и OpenMP,
которая позволяет решать задачу Коши\cite{koshi} для уравнения теплопроводности
в двумерной прямоугольной области. ( Описанно формулами \eqref{heat-transfer}.)

\begin{equation}
\label{heat-transfer}
\left\{ 
    \begin{array}{l}
        \frac{\partial u(x,y,t)}{\partial t} = 
            a^2 \left( 
                  \frac{\partial^2 u(x,y)}{\partial x^2} +  
                  \frac{\partial^2 u(x,y)}{\partial y^2} 
                \right) +
                  heat(x,y,t) \\
        u(x,y,t_0) = init(x,y) \\
        \left. u(x,y,t) \right|_{x,y \in G} = border(x,y,t) \\
    \end{array}
\right.    
\end{equation}

Данная задача, заменяется разностной задачей на сетке \eqref{diff-heat}.
\begin{equation}
\label{diff-heat}
    \begin{array}{l}
                \frac
                { 
                    u(x,y,t+\Delta{t}) - u(x,y,t) 
                }
                {\Delta{t}} = \\
            =   a^2  
            
            \left(
                    \frac 
                    {
                     u(x+\Delta{x},y,t) - 2 \cdot u(x,y,t)  + u(x - \Delta{x},y,t)
                    }
                    {2 \cdot \Delta{x}^2 } + 

                    \frac 
                    {
                        u(x,y + \Delta{y},t) - 2 \cdot u(x,y,t) + u(x,y - \Delta{y},t)
                    }
                    {2 \cdot \Delta{y}^2}
            \right)
                + \\
                + heat(x,y,t)
          \end{array}
\end{equation}

Предполагается, что область делится на набор равномерных
отрезков по каждому изменению с шагами $\Delta{x}$ и $\Delta{y}$. Мы знаем состояние в 
начальнй момент времени, задаётся функцией $init(x,y)$, которую можно вычислить дискретно,
тоесть задать начальное состояние системы матрицей. Нам известен закон поведения 
границы области во времени. Это означает, что в тот момент, когда мы будем подходить к границе 
в программе, для точек границы необходимо будет вызвать функцию $border(x,y,t)$, внутри которой
значение можно будет вычислить по некоторому непрерывному закону. 

\section{Параметры передаваемые программному коду}

Для обеспечения моделирования распространения тепла программе необходимо передать следующие параметры.
\begin{itemize}
    \item {\bf шаг по измерению $X$ } -- собственно $\Delta{x}$.
    \item {\bf шаг по измерению $Y$ } -- собственно $\Delta{y}$.
    \item {\bf шаг по времени } -- собственно $\Delta{t}$.
    \item {\bf промежуток сброса в файл} -- через какие промежутки времени в терминах времени модели, 
                                            а не астрономического времени машины на которой проводится 
                                            математическое моделирование, необходимо сбрасывать результаты
                                            в файл.
   \item {\bf файл с начальной матрицей} -- начальное состояние системы.
   \item{\bf  префикс имён файлов с результатами} -- имя файла задаётся в формате 
         \textit{префикс\_модельное\_время}.
   \item {\bf число потоков} -- Число OpenMP потоков, которое придётся на один узел кластера.
   \item {\bf функция для рабты с границей } -- выбор функции для вычисления границы из некоторого списка.
   \item {\bf функция источников тепла} -- выбор одного из вариантов для функции $heat(x,y,t)$
\end{itemize}
Все параметры можно поместить в специальный конфигурационный файл, а так же можно задать параметры по умолчанию.
Если параметрв передаются не через конфигурационный файл, то весьма желательно, чтобы для их разбора была 
использована функция getopt или getoptlong.

\section{Список программ и файлов}

Итак, необходимо написать:
\begin{enumerate}
    \item последовательную программу
    \item параллельную  MPI программу, которая допускает запуск себя в режиме, 
          когда есть только многопоточность и MPI не задействован.
    \item набор файлов, в которых находятся реализации функций \textit{heat} и 
          \textit{border}. (Предусмотреть возможность указания функций, которые обнуляют
          границу и не вносят дополнительного тепла). В идеале реализовать их подгрузку как 
          .so/.dll объектов с использованием функции dlopen, но на Bluegene не будет работать. 
    \item Makefile, которым всё это компилируется и, возможно, запускаются 
          тесты на вычислительном кластере.
\end{enumerate}

\section{Эксперименты и графики}

Собственно построить те же графики, что и в предыдущих заданиях, однако здесь необходимо задействовать
MPI-IO, и построить графики с учётом накладных расходов связанных с записью в файл.





\begin{thebibliography}{50}
    \bibitem{koshi} Страничка в википедии про уравнение теплопроводности: 
        \url{https://en.wikipedia.org/wiki/Heat_equation}
\end{thebibliography}

\end{document}
